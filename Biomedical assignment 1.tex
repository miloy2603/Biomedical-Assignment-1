\documentclass[a4paper,12pt]{report}
\usepackage{graphicx}
\title{TWO PAGE WRITE-UP ON 5 MEDICAL DEVICES}
\author{NAME :- MILOY KUMAR MANDAL\\ROLL NO :- 21111032\\BRANCH :- BIOMEDICAL ENGINEERING\\}
\date{\today}
\clearpage
\begin{document}
\maketitle
\pagenumbering{roman}
\tableofcontents
\newpage
\pagenumbering{arabic}

\chapter{CT SCANNER}
\section{What is a computed tomography (CT) scan?}
The term “computed tomography”, or CT, refers to a computerized x-ray imaging procedure in which a narrow beam of x-rays is aimed at a patient and quickly rotated around the body, producing signals that are processed by the machine’s computer to generate cross-sectional images—or “slices”—of the body. These slices are called tomographic images and contain more detailed information than conventional x-rays. Once a number of successive slices are collected by the machine’s computer, they can be digitally “stacked” together to form a three-dimensional image of the patient that allows for easier identification and location of basic structures as well as possible tumors or abnormalities.\\ 
\section{History}
The first commercially available CT scanner was created by British engineer Godfrey Hounsfield of EMI Laboratories in 1972. He co-invented the technology with physicist Dr. Allan Cormack. Both researchers were later on jointly awarded the 1979 Nobel Prize in Physiology and Medicine. By 1981, Hounsfield was knighted and became Sir Godfrey Hounsfield.\\
\section{How it works?}
Unlike a conventional x-ray—which uses a fixed x-ray tube—a CT scanner uses a motorized x-ray source that rotates around the circular opening of a donut-shaped structure called a gantry. During a CT scan, the patient lies on a bed that slowly moves through the gantry while the x-ray tube rotates around the patient, shooting narrow beams of x-rays through the body. Instead of film, CT scanners use special digital x-ray detectors, which are located directly opposite the x-ray source. As the x-rays leave the patient, they are picked up by the detectors and transmitted to a computer.Each time the x-ray source completes one full rotation, the CT computer uses sophisticated mathematical techniques to construct a 2D image slice of the patient. The thickness of the tissue represented in each image slice can vary depending on the CT machine used, but usually ranges from 1-10 millimeters. When a full slice is completed, the image is stored and the motorized bed is moved forward incrementally into the gantry. The x-ray scanning process is then repeated to produce another image slice. This process continues until the desired number of slices is collected.Image slices can either be displayed individually or stacked together by the computer to generate a 3D image of the patient that shows the skeleton, organs, and tissues as well as any abnormalities the physician is trying to identify. This method has many advantages including the ability to rotate the 3D image in space or to view slices in succession, making it easier to find the exact place where a problem may be located.
\section{What are its risks?}
CT scans can diagnose possibly life-threatening conditions such as hemorrhage, blood clots, or cancer. An early diagnosis of these conditions could potentially be life-saving. However, CT scans use x-rays, and all x-rays produce ionizing radiation. Ionizing radiation has the potential to cause biological effects in living tissue. This is a risk that increases with the number of exposures added up over the life of an individual. However, the risk of developing cancer from radiation exposure is generally small.

A CT scan in a pregnant woman poses no known risks to the baby if the area of the body being imaged isn’t the abdomen or pelvis. In general, if imaging of the abdomen and pelvis is needed, doctors prefer to use exams that do not use radiation, such as MRI or ultrasound. However, if neither of those can provide the answers needed, or there is an emergency or other time constraint, CT may be an acceptable alternative imaging option.

In some patients, contrast agents may cause allergic reactions, or in rare cases, temporary kidney failure. IV contrast agents should not be administered to patients with abnormal kidney function since they may induce a further reduction of kidney function, which may sometimes become permanent.  

Children are more sensitive to ionizing radiation and have a longer life expectancy and, thus, a higher relative risk for developing cancer than adults. Parents may want to ask the technologist or doctor if their machine settings have been adjusted for children.
\begin{figure}
\graphicspath{{C:\Users\Lenovo}}
\centering
\includegraphics[scale=1] {ctscanner.jpg}
\caption{MRI machine}
\end{figure}



\chapter{Viscometer}
\section{What is Viscometer?}
A viscometer (also called viscosimeter) is an instrument used to measure the viscosity of a fluid. For liquids with viscosities which vary with flow conditions, an instrument called a rheometer is used. Thus, a rheometer can be considered as a special type of viscometer.Viscometers only measure under one flow condition.
\section{Theory behind Viscometer}
In general, either the fluid remains stationary and an object moves through it, or the object is stationary and the fluid moves past it. The drag caused by relative motion of the fluid and a surface is a measure of the viscosity. The flow conditions must have a sufficiently small value of Reynolds number for there to be laminar flow.

At 20 °C, the dynamic viscosity (kinematic viscosity × density) of water is 1.0038 mPa·s and its kinematic viscosity (product of flow time × factor) is 1.0022 mm2/s. These values are used for calibrating certain types of viscometers.
\section{Uses of Viscometer}
Many industries measure viscosity, though the biggest user is the quality control department using single-point measurement. Research scientists also use viscometers to see how a material reacts to being sheared. The task at hand determines the kind of viscometer to use— different viscometers measure different magnitudes of viscosity and different changes in it. According to one expert, the most important factor to consider when buying a viscometer is robustness, even if users have to give up some sensitivity.
\begin{figure}
\graphicspath{{C:\Users\Lenovo}}
\centering
\includegraphics[scale=1] {viscometer.jpg}
\caption{A viscometer}
\end{figure}
\section{Pharmaceutical and Biomedical application}
Viscometry or viscosity method was one of the first methods used for determining the MW of polymers. In this method, the viscosity of polymer solution is measured, and the simplest method used is capillary viscometry by using the Ubbelohde U-tube viscometer. The Ubbelohde capillary viscometer is used to determine the polymer’s solution viscosity. The process includes introducing the polymer solution into the reservoir of the viscometer, and then aspirating to the upper bulb. Afterward, the admission of the air causes the polymer solution to flow down in the capillary by gravity. To determine the average MW (Mn) by this technique, the process should be performed for pure polymer solution and solvents respectively. The pressure drop  varies based on the viscosity  of the solutions due to the fact that the polymer solution that flows through the capillary follows Poiseuille’s law for laminar flow



\chapter{Computed Radiography (CR) system}
\section{Health problem addressed}
This device is an image digitization system designed to acquire and digitize x-ray images from image storage phosphor plates. The patient positioning and imaging techniques used in CR imaging are identical to those used in conventional radiography. Clinical applications include all radiographic examinations performed by conventional table systems (e.g., pediatric, skeletal, abdominal, urologic imaging) and portable systems.
\section{Priciples of operation}
Imaging plates are inserted in a radiographic table’s cassette holder and images are acquired using the x-ray system. When exposed to x-rays, electrons in the phosphor plate are excited into a higher energy state, forming a latent image. An image reader scans the phosphor plate with a laser spot. When the trapped electrons absorb the laser energy, they emit light as they return to their ground state. This light is collected by a light guide and transmitted to a photomultiplier tube, which produces an analog electrical signal that is amplified, converted to a digital signal, and digitally stored. The plate can be reused after it is exposed to an erasing light that removes residual radiation.


\begin{figure}
\graphicspath{{C:\Users\Lenovo}}
\centering
\includegraphics[scale=1] {cr.jpg}
\caption{Computed Radiography Machine}
\end{figure}  

\begin{figure}
\graphicspath{{C:\Users\Lenovo}}
\centering
\includegraphics[scale=0.6] {cr1.jpg}
\caption{Computed Radiography System}
\end{figure}  


\section{Reported problems}
Most reported problems involve the condition of imaging plates and pose little to no direct danger to the patient. Plates can be damaged by careless handling and are expensive to replace. Some readers have brushes or fans that automatically clear dust off plates, which helps to prevent scratches. Cracks or scratches would render the plate unusable for imaging. The light guide should be cleaned as part of routine maintenance (dirt accumulation can affect image quality).

\section{Product description}
A CR system consists of an image reader/digitizer, cassettes containing imaging receptors (photostimulable-phosphor plates), a computer console or workstation, software, monitors, and a printer. Single-plate readers (each cassette is loaded manually and read separately) and multiple-plate readers (multiple plates—up to 10—can be stacked and loaded automatically) are available.



\chapter{ECG}
\section{What is ECG?}
An electrocardiogram (ECG) is a simple test that can be used to check your heart's rhythm and electrical activity.Sensors attached to the skin are used to detect the electrical signals produced by your heart each time it beats.These signals are recorded by a machine and are looked at by a doctor to see if they're unusual.An ECG may be requested by a heart specialist (cardiologist) or any doctor who thinks you might have a problem with your heart, including your GP.The test can be carried out by a specially trained healthcare professional at a hospital, a clinic or at your GP surgery.Despite having a similar name, an ECG isn't the same as an echocardiogram, which is a scan of the heart.
\section{History}
Dr. Willem Einthoven, a Dutch physiologist inspired by the work of Waller, refined the capillary electrometer even further and was able to demonstrate five deflections which he named ABCDE . To adjust for inertia in the capillary system, he implemented a mathematical correction, which resulted in the curves that we see today. Following the mathematical tradition established by Descartes , he used the terminal part of alphabet series (PQRST) to name these deflections. The term ‘electrocardiogram’ used to describe these wave forms was first coined by Einthoven at the Dutch Medical Meeting of 1893. In 1901, he successfully developed a new string galvanometer with very high sensitivity, which he used in his electrocardiograph. His device weighed 600 pounds
\section{Its Uses}
An ECG is often used alongside other tests to help diagnose and monitor conditions affecting the heart.

It can be used to investigate symptoms of a possible heart problem, such as chest pain, palpitations (suddenly noticeable heartbeats), dizziness and shortness of breath.

An ECG can help detect:\\

\textbf{Arrhythmias} – where the heart beats too slowly, too quickly, or irregularly.\\
\textbf{Coronary heart disease}– where the heart's blood supply is blocked or interrupted by a build-up of fatty substances.\\
\textbf{Heart attacks} – where the supply of blood to the heart is suddenly blocked.\\
\textbf{Cardiomyopathy} – where the heart walls become thickened or enlarged.\\

A series of ECGs can also be taken over time to monitor a person already diagnosed with a heart condition or taking medication known to potentially affect the heart.

\begin{figure}
\graphicspath{{C:\Users\Lenovo}}
\centering
\includegraphics[scale=1] {ecg.jpg}
\caption{Electrocardiogram}
\end{figure}

\chapter{ULTRACENTRIFUGE}
\section{What is ultracentrifuge?}
The ultracentrifuge is a centrifuge optimized for spinning a rotor at very high speeds, capable of generating acceleration as high as 1000000 g .There are two kinds of ultracentrifuges, the preparative and the analytical ultracentrifuge. Both classes of instruments find important uses in molecular biology, biochemistry, and polymer science.

\section{History} 

In 1946, Pickels cofounded Spinco (Specialized Instruments Corp.) to market analytical and preparative ultracentrifuges based on his design. Pickels considered his design to be too complicated for commercial use and developed a more easily operated, “foolproof” version. But even with the enhanced design, sales of analytical centrifuges remained low, and Spinco almost went bankrupt. The company survived by concentrating on sales of preparative ultracentrifuge models, which were becoming popular as workhorses in biomedical laboratories.In 1949, Spinco introduced the Model L, the first preparative ultracentrifuge to reach a maximum speed of 40,000 rpm. In 1954, Beckman Instruments (later Beckman Coulter) purchased the company, forming the basis of its Spinco centrifuge division.

\section{Instrumentation}

Ultracentrifuges are available with a wide variety of rotors suitable for a great range of experiments. Most rotors are designed to hold tubes that contain the samples. Swinging bucket rotors allow the tubes to hang on hinges so the tubes reorient to the horizontal as the rotor initially accelerate. Fixed angle rotors are made of a single block of material and hold the tubes in cavities bored at a predetermined angle. Zonal rotors are designed to contain a large volume of sample in a single central cavity rather than in tubes. Some zonal rotors are capable of dynamic loading and unloading of samples while the rotor is spinning at high speed.

Preparative rotors are used in biology for pelleting of fine particulate fractions, such as cellular organelles (mitochondria, microsomes, ribosomes) and viruses. They can also be used for gradient separations, in which the tubes are filled from top to bottom with an increasing concentration of a dense substance in solution. Sucrose gradients are typically used for separation of cellular organelles. Gradients of caesium salts are used for separation of nucleic acids. After the sample has spun at high speed for sufficient time to produce the separation, the rotor is allowed to come to a smooth stop and the gradient is gently pumped out of each tube to isolate the separated components.


\begin{figure}
\graphicspath{{C:\Users\Lenovo}}
\centering
\includegraphics[scale=1] {ultracentrifuge.jpg}
\caption{Ultracentrifuge}
\end{figure}


\section{Hazards}

The tremendous rotational kinetic energy of the rotor in an operating ultracentrifuge makes the catastrophic failure of a spinning rotor a serious concern, as it can explode spectacularly. Rotors conventionally have been made from high strength-to-weight metals such as aluminum or titanium. The stresses of routine use and harsh chemical solutions eventually cause rotors to deteriorate. Proper use of the instrument and rotors within recommended limits and careful maintenance of rotors to prevent corrosion and to detect deterioration is necessary to mitigate this risk.

More recently some rotors have been made of lightweight carbon fiber composite material, which are up to 60PERCENT  lighter, resulting in faster acceleration/deceleration rates. Carbon fiber composite rotors also are corrosion-resistant, eliminating a major cause of rotor failure.



\end{document}